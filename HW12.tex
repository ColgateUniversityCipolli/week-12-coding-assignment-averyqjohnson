\documentclass{article}\usepackage[]{graphicx}\usepackage[]{xcolor}
% maxwidth is the original width if it is less than linewidth
% otherwise use linewidth (to make sure the graphics do not exceed the margin)
\makeatletter
\def\maxwidth{ %
  \ifdim\Gin@nat@width>\linewidth
    \linewidth
  \else
    \Gin@nat@width
  \fi
}
\makeatother

\definecolor{fgcolor}{rgb}{0.345, 0.345, 0.345}
\newcommand{\hlnum}[1]{\textcolor[rgb]{0.686,0.059,0.569}{#1}}%
\newcommand{\hlsng}[1]{\textcolor[rgb]{0.192,0.494,0.8}{#1}}%
\newcommand{\hlcom}[1]{\textcolor[rgb]{0.678,0.584,0.686}{\textit{#1}}}%
\newcommand{\hlopt}[1]{\textcolor[rgb]{0,0,0}{#1}}%
\newcommand{\hldef}[1]{\textcolor[rgb]{0.345,0.345,0.345}{#1}}%
\newcommand{\hlkwa}[1]{\textcolor[rgb]{0.161,0.373,0.58}{\textbf{#1}}}%
\newcommand{\hlkwb}[1]{\textcolor[rgb]{0.69,0.353,0.396}{#1}}%
\newcommand{\hlkwc}[1]{\textcolor[rgb]{0.333,0.667,0.333}{#1}}%
\newcommand{\hlkwd}[1]{\textcolor[rgb]{0.737,0.353,0.396}{\textbf{#1}}}%
\let\hlipl\hlkwb

\usepackage{framed}
\makeatletter
\newenvironment{kframe}{%
 \def\at@end@of@kframe{}%
 \ifinner\ifhmode%
  \def\at@end@of@kframe{\end{minipage}}%
  \begin{minipage}{\columnwidth}%
 \fi\fi%
 \def\FrameCommand##1{\hskip\@totalleftmargin \hskip-\fboxsep
 \colorbox{shadecolor}{##1}\hskip-\fboxsep
     % There is no \\@totalrightmargin, so:
     \hskip-\linewidth \hskip-\@totalleftmargin \hskip\columnwidth}%
 \MakeFramed {\advance\hsize-\width
   \@totalleftmargin\z@ \linewidth\hsize
   \@setminipage}}%
 {\par\unskip\endMakeFramed%
 \at@end@of@kframe}
\makeatother

\definecolor{shadecolor}{rgb}{.97, .97, .97}
\definecolor{messagecolor}{rgb}{0, 0, 0}
\definecolor{warningcolor}{rgb}{1, 0, 1}
\definecolor{errorcolor}{rgb}{1, 0, 0}
\newenvironment{knitrout}{}{} % an empty environment to be redefined in TeX

\usepackage{alltt}
\usepackage[margin=1.0in]{geometry} % To set margins
\usepackage{amsmath}  % This allows me to use the align functionality.
                      % If you find yourself trying to replicate
                      % something you found online, ensure you're
                      % loading the necessary packages!
\usepackage{amsfonts} % Math font
\usepackage{fancyvrb}
\usepackage{hyperref} % For including hyperlinks
\usepackage[shortlabels]{enumitem}% For enumerated lists with labels specified
                                  % We had to run tlmgr_install("enumitem") in R
\usepackage{float}    % For telling R where to put a table/figure
\usepackage{natbib}        %For the bibliography
\bibliographystyle{apalike}%For the bibliography
\IfFileExists{upquote.sty}{\usepackage{upquote}}{}
\begin{document}


\begin{enumerate}
%%%%%%%%%%%%%%%%%%%%%%%%%%%%%%%%%%%%%%%%%%%%%%%%%%%%%%%%%%%%%%%%%%%%%%%%%%%%%%%%
%%%%%%%%%%%%%%%%%%%%%%%%%%%%%%%%%%%%%%%%%%%%%%%%%%%%%%%%%%%%%%%%%%%%%%%%%%%%%%%%
% Question 1
%%%%%%%%%%%%%%%%%%%%%%%%%%%%%%%%%%%%%%%%%%%%%%%%%%%%%%%%%%%%%%%%%%%%%%%%%%%%%%%%
%%%%%%%%%%%%%%%%%%%%%%%%%%%%%%%%%%%%%%%%%%%%%%%%%%%%%%%%%%%%%%%%%%%%%%%%%%%%%%%%
\item A group of researchers is running an experiment over the course of 30 months, 
with a single observation collected at the end of each month. Let $X_1, ..., X_{30}$
denote the observations for each month. From prior studies, the researchers know that
\[X_i \sim f_X(x),\]
but the mean $\mu_X$ is unknown, and they wish to conduct the following test
\begin{align*}
H_0&: \mu_X = 0\\
H_a&: \mu_X > 0.
\end{align*}
At month $k$, they have accumulated data $X_1, ..., X_k$ and they have the 
$t$-statistic
\[T_k = \frac{\bar{X} - 0}{S_k/\sqrt{n}}.\]
The initial plan was to test the hypotheses after all data was collected (at the 
end of month 30), at level $\alpha=0.05$. However, conducting the experiment is 
expensive, so the researchers want to ``peek" at the data at the end of month 20 
to see if they can stop it early. That is, the researchers propose to check 
whether $t_{20}$ provides statistically discernible support for the alternative. 
If it does, they will stop the experiment early and report support for the 
researcher's alternative hypothesis. If it does not, they will continue to month 
30 and test whether $t_{30}$ provides statistically discernible support for the
alternative.

\begin{enumerate}
  \item What values of $t_{20}$ provide statistically discernible support for the
  alternative hypothesis?
  
\begin{knitrout}\scriptsize
\definecolor{shadecolor}{rgb}{0.969, 0.969, 0.969}\color{fgcolor}\begin{kframe}
\begin{alltt}
\hlcom{# (a) t-val for statistically discernible support for t20}
\hlcom{# gives t val at 95th percentile }
\hldef{(val_t20} \hlkwb{<-} \hlkwd{qt}\hldef{(}\hlnum{0.95}\hldef{,} \hlkwc{df}\hldef{=}\hlnum{19}\hldef{))}
\end{alltt}
\begin{verbatim}
## [1] 1.729133
\end{verbatim}
\end{kframe}
\end{knitrout}

To determine whether the researchers should stop the experiment at month $20$,
we calculated the critical value of the one-sided t-test at the 5\% significance
level with 19 degrees of freedom (since $n=20$). This gave a critical value of
$1.729$, so if $t_{20} > 1.729$, the researchers would reject $H_{0}$ and stop early.
  
  \item What values of $t_{30}$ provide statistically discernible support for the
  alternative hypothesis?
  
\begin{knitrout}\scriptsize
\definecolor{shadecolor}{rgb}{0.969, 0.969, 0.969}\color{fgcolor}\begin{kframe}
\begin{alltt}
  \hlcom{# (b) t-val for statistically discernible support for t30}
  \hlcom{# gives t val at 95th percentile }
  \hldef{(val_t30} \hlkwb{<-} \hlkwd{qt}\hldef{(}\hlnum{0.95}\hldef{,} \hlnum{29}\hldef{))}
\end{alltt}
\begin{verbatim}
## [1] 1.699127
\end{verbatim}
\end{kframe}
\end{knitrout}
  
If the experiment continues to month 30, the researchers use the full sample to compute a new test statistic and compare it to the critical value for a t-test with 29 degrees of freedom (since $n=30$). This gave a critical value of $1.699$,
so if $t_{30} > 1.699$, they would reject $H_{0}$ at that point.


  \item Suppose $f_X(x)$ is a Laplace distribution with $a=0$ and $b=4.0$.
  Conduct a simulation study to assess the Type I error rate of this approach.\\
  \textbf{Note:} You can use the \texttt{rlaplace()} function from the \texttt{VGAM}
  package for \texttt{R} \citep{VGAM}.
  
\begin{knitrout}\scriptsize
\definecolor{shadecolor}{rgb}{0.969, 0.969, 0.969}\color{fgcolor}\begin{kframe}
\begin{alltt}
\hlcom{# (c) simulation to estimate type 1 error}

\hlkwd{library}\hldef{(VGAM)}
\end{alltt}


{\ttfamily\noindent\itshape\color{messagecolor}{\#\# Loading required package: stats4}}

{\ttfamily\noindent\itshape\color{messagecolor}{\#\# Loading required package: splines}}\begin{alltt}
\hldef{simulations} \hlkwb{<-} \hlnum{10000}
\hldef{alpha} \hlkwb{<-} \hlnum{0.05}

\hldef{type_1_error_count} \hlkwb{<-} \hlnum{0}

\hlkwa{for} \hldef{(i} \hlkwa{in} \hlnum{1}\hlopt{:}\hldef{simulations) \{}
  \hlcom{# generate data from Laplace}
  \hldef{data} \hlkwb{<-} \hlkwd{rlaplace}\hldef{(}\hlnum{30}\hldef{,} \hlkwc{location}\hldef{=}\hlnum{0}\hldef{,} \hlkwc{scale}\hldef{=}\hlnum{4}\hldef{)}

  \hlcom{# perform t test}
  \hldef{t20_result} \hlkwb{<-} \hlkwd{t.test}\hldef{(data[}\hlnum{1}\hlopt{:}\hlnum{20}\hldef{],} \hlkwc{mu}\hldef{=}\hlnum{0}\hldef{)}
  \hldef{t20} \hlkwb{<-} \hldef{t20_result}\hlopt{$}\hldef{statistic}

  \hldef{t30_result} \hlkwb{<-} \hlkwd{t.test}\hldef{(data,} \hlkwc{mu}\hldef{=}\hlnum{0}\hldef{)}
  \hldef{t30} \hlkwb{<-} \hldef{t30_result}\hlopt{$}\hldef{statistic}

  \hlcom{# check if we would reject the null at month 20}
  \hlcom{# do the t-stats exceed the corresponding critical values?}
    \hlcom{# if they do, reject the null}
  \hlkwa{if} \hldef{(t20} \hlopt{>} \hldef{val_t20) \{}
    \hldef{type_1_error_count} \hlkwb{<-} \hldef{type_1_error_count} \hlopt{+} \hlnum{1}
    \hlcom{# if this doesnt reject, then we check t30}
  \hldef{\}} \hlkwa{else if} \hldef{(t30} \hlopt{>} \hldef{val_t30) \{}
    \hldef{type_1_error_count} \hlkwb{<-} \hldef{type_1_error_count} \hlopt{+} \hlnum{1}
  \hldef{\}}
\hldef{\}}

\hlcom{# estimate type 1 error rate}
\hldef{(type_1_error_rate} \hlkwb{<-} \hldef{type_1_error_count} \hlopt{/} \hldef{simulations)}
\end{alltt}
\begin{verbatim}
## [1] 0.0756
\end{verbatim}
\end{kframe}
\end{knitrout}
  
To examine the impact of this strategy, we simulated 10,000 experiments under
the null hypothesis, where data were drawn from a Laplace distribution with mean zero and scale 4. In each simulation, we computed $t_{20}$. If it was significant, we stopped and counted it as rejection. Otherwise, we computed $t_{30}$ and checked again. This procedure gave a type one error rate of approximately 0.073, or 7.3\%, which is noticeably higher than the intended 5\%. This confirms that checking twice without adjusting for it increases the overall chance of making a
Type I error.
  
  \item \textbf{Optional Challenge:} Can you find a value of $\alpha<0.05$ that yields a 
  Type I error rate of 0.05?
\end{enumerate}
%%%%%%%%%%%%%%%%%%%%%%%%%%%%%%%%%%%%%%%%%%%%%%%%%%%%%%%%%%%%%%%%%%%%%%%%%%%%%%%%
%%%%%%%%%%%%%%%%%%%%%%%%%%%%%%%%%%%%%%%%%%%%%%%%%%%%%%%%%%%%%%%%%%%%%%%%%%%%%%%%
% Question 2
%%%%%%%%%%%%%%%%%%%%%%%%%%%%%%%%%%%%%%%%%%%%%%%%%%%%%%%%%%%%%%%%%%%%%%%%%%%%%%%%
%%%%%%%%%%%%%%%%%%%%%%%%%%%%%%%%%%%%%%%%%%%%%%%%%%%%%%%%%%%%%%%%%%%%%%%%%%%%%%%%
  \item Perform a simulation study to assess the robustness of the $T$ test. 
  Specifically, generate samples of size $n=15$ from the Beta(10,2), Beta(2,10), 
  and Beta(10,10) distributions and conduct the following hypothesis tests against 
  the actual mean for each case (e.g., $\frac{10}{10+2}$, $\frac{2}{10+2}$, and 
  $\frac{10}{10+10}$). 
  \begin{enumerate}
    \item What proportion of the time do we make an error of Type I for a
    left-tailed test?

    \item What proportion of the time do we make an error of Type I for a
    right-tailed test?

    \item What proportion of the time do we make an error of Type I for a
    two-tailed test?
    
    
    \item How does skewness of the underlying population distribution effect
    Type I error across the test types?


  \end{enumerate}
%%%%%%%%%%%%%%%%%%%%%%%%%%%%%%%%%%%%%%%%%%%%%%%%%%%%%%%%%%%%%%%%%%%%%%%%%%%%%%%%
%%%%%%%%%%%%%%%%%%%%%%%%%%%%%%%%%%%%%%%%%%%%%%%%%%%%%%%%%%%%%%%%%%%%%%%%%%%%%%%%
% End Document
%%%%%%%%%%%%%%%%%%%%%%%%%%%%%%%%%%%%%%%%%%%%%%%%%%%%%%%%%%%%%%%%%%%%%%%%%%%%%%%%
%%%%%%%%%%%%%%%%%%%%%%%%%%%%%%%%%%%%%%%%%%%%%%%%%%%%%%%%%%%%%%%%%%%%%%%%%%%%%%%%
\end{enumerate}
\bibliography{bibliography}
\end{document}
